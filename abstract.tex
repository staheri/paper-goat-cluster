The use of increasing levels of parallelism and concurrency in system
design---especially in a feature-rich language such as Go---demands
effective concurrency debugging techniques that are easy to deploy in practice.
%
We present GoAT, a combined static and dynamic concurrency analysis
and debugging approach
that maximally takes advantage of Go's existing tooling infrastructure.
%
Given the nascent state of Go debugging, our approach builds
a debugging approach by deriving insights from recent Go bug benchmark
suites.
%
Key ideas in GoAT include static instrumentation of yield-points
at all concurrency sites followed by runtime selection of yields with
a user-specified probability.
%
The approach in GoAT can coexist with existing dynamic Go verification tooling
for memory leak detection and data race checking while adding significant
new capabilities toward concurrency debugging.
%
Our evaluation on 68 curated real-world bug scenarios
demonstrates that GoAT is significantly effective in detecting
rare bugs, and its schedule perturbation method based on schedule
yielding detects these bugs within about five yields.
%
These results together with the ease of deploying GoAT on real-world
Go programs holds significant promise in field-debugging of Go programs.
