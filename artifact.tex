


%%%%%%%%%%%%%%%%%%%%%%%%%%%%%%%%%%%%%%%%%%%%%%%%%%%%%%%%%%%%%%%%%%%%%
\subsection{Abstract}

\goat is a combined static and dynamic concurrency testing and analysis tool that facilitates the process of debugging for real-world Go programs.
%
The workflow of \goat includes
1) automated dynamic tracing to capture the behavior of concurrency primitives,
2) systematic schedule space exploration to accelerate the bug occurrence
and 3) deadlock detection with supplementary visualizations and reports.
\goat also proposes a set of coverage requirements that characterize the dynamic behavior of concurrency primitives and provide metrics to measure the quality of tests.

All of the above are done through \textsf{goatlib} (tracing API) and tuning parameters such as global deadlock timeout, visualization grain, and the number of run-time processes.

\goat's source is available at \textsf{https://github.com/staheri/goat.git}.
We are working on a docker version of goat to make it available for schedule testing through test packages.

\subsection{Artifact check-list (meta-information)}

{\em Obligatory. Use just a few informal keywords in all fields applicable to your artifacts
and remove the rest. This information is needed to find appropriate reviewers and gradually
unify artifact meta information in Digital Libraries.}

{\small
\begin{itemize}
  \item {\bf Algorithm: }Dynamic tracing, testing coverage, schedule exploration, debugging, visualization, and software analysis.
  \item {\bf Program: } Automated source instrumentation that utilizes a custom Go run-time to collect traces containing concurrent events. Traces are automatically analyzed to discover the program's behavior, detect deadlocks, and measure concurrent coverage metrics.
  \item {\bf Data set: } We present the evaluation of \goat and its comparison with existing dynamic detectors on GoBench \cite{yuan-gobench-cgo21} concurrency bugs. This paper's tables and figures are obtained from the data available in (ZENODO) and are reproducible using \goat software.
  \item {\bf Run-time environment: } Using a one-time patch, we rebuild the Go run-time from version 1.15.6 to capture concurrency events. The instruction on how to build the run-time is available in README in (ZENODO)
  \item {\bf Hardware: } \goat is built on a Linux x86-64 machine. However, it will work on other architectures.
  \item {\bf Execution: } \goat's executable
  \item {\bf Metrics: } Concurrency coverage, synchronization coverage.
  \item {\bf Output: } Deadlock report, coverage requirements and tables, execution visualization.
  \item {\bf How much disk space required (approximately)?: } \goat itself is less than 1 MB. However, for extensive experiments and schedule exploration, depending on the number of iterations and size of the program, 10 GB is the maximum amount of data (instrumented executables, traces, reports, visualizations) generated by \goat.
  \item {\bf How much time is needed to prepare workflow (approximately)?: } Following the instructions in the README, it should take less than 10 minutes to build \goat.
  \item {\bf How much time is needed to complete experiments (approximately)?: } Producing our experimental results presented in this paper took less than 12 hours on a machine with 64 GB memory and 16 processors.
  \item {\bf Publicly available?: } \goat source is publicly available at \textsf{https://github.com/staheri/goat.git} and Zenodo website at \href{https://doi.org/10.5281/zenodo.5519040}{https://doi.org/10.5281/zenodo.5519040} \cite{goat_artifact}.
  \item {\bf Archived (provide DOI)?: } 10.5281/zenodo.5519040
\end{itemize}

%%%%%%%%%%%%%%%%%%%%%%%%%%%%%%%%%%%%%%%%%%%%%%%%%%%%%%%%%%%%%%%%%%%%%
\subsection{Description}

\subsubsection{How to access}
The artifact source is publicly available and constructible at \textsf{https://github.com/staheri/goat.git} and Zenodo website at \href{https://doi.org/10.5281/zenodo.5519040}{https://doi.org/10.5281/zenodo.5519040} \cite{goat_artifact}.

\subsubsection{Hardware dependencies}
\goat is built on a Linux x86-64 machine. However, it will work on other architectures.

\subsubsection{Software dependencies}
Go version 1.15.6 is required.
\subsubsection{Data sets}
We present the evaluation of \goat, and its comparison with existing dynamic detectors on GoBench \cite{yuan-gobench-cgo21} concurrency bugs. The data presented in this paper (summarized in table \ref{tab:comparison}) is obtained from the data available in (ZENODO) and is also reproducible using \goat software.

%%%%%%%%%%%%%%%%%%%%%%%%%%%%%%%%%%%%%%%%%%%%%%%%%%%%%%%%%%%%%%%%%%%%%
\subsection{Installation}

GoAT is working in a custom run-time based on version \textbf{1.15.6} of Golang.
Lets say the original Go installation is under \textsf{/usr/local/go}. Then the commands in listing \ref{appendix_first} statements must be executed to build the \goat's runtime and resolve its dependencies:

\begin{listing}[]
\begin{minted}
[
fontsize=\footnotesize,
linenos=false,
escapeinside=||,
bgcolor=lightgray,
breaklines
]
{bash}
sudo -i
mv /usr/local/go /usr/local/go-orig
ln -nsf /usr/local/go-orig /usr/local/go

# download fresh version 1.15.6
wget https://golang.org/dl/go1.15.6.linux -amd64.tar.gz
mkdir -p /usr/local/go-goat
tar -xzvf go1.15.6.linux-amd64.tar.gz -C /usr/local/go-goat

# set environment variables
export GOPATH=$HOME/gopath
mkdir $GOPATH
export GOROOT=/usr/local/go
export PATH=$PATH:$GOROOT/bin:$GOPATH/bin
export GOATWS=$HOME/goatws
mkdir -p $GOATWS
export GOATTO=30
export GOATMAXPROCS=4

# path safety
go get github.com/staheri/goat

# build the runtime
cd /usr/local/go-goat/go
patch -p2 -d src/  < $GOPATH/src/github.com/staheri/goat/ go1.15.6_goat_june15.patch
cd src/
export GOROOT_BOOTSTRAP=/usr/local/go-orig
./make.bash
ln -nsf /usr/local/go-goat/go /usr/local/go
cd $GOPATH/src/github.com/staheri/goat
go build -o $GOPATH/bin/goat
\end{minted}
\caption{Installation commands}
\label{appendix_first}
\end{listing}

\subsection{Experiment workflow}


\begin{listing}[]
\begin{minted}
[
fontsize=\footnotesize,
linenos=false,
escapeinside=||,
breaklines
]
{bash}

Usage of bin/goat-single:
  -cov
        Include coverage report in evaluation
  -d int
        Number of delays
  -eval_conf string
        Config file with benchmark paths in it
  -freq int
        Frequency of executions (default 1)
  -path string
        Target folder
  -race
        Enable race detection
\end{minted}
\caption{\goat's help message}
\label{appendix_second}
\end{listing}
Listing \ref{appendix_second} shows the output of \goat's help message that illustrate its workflow.
For example, for collecting traces from the execution of ``CodeBenchmark/goBench\_goat/defSel/goatDefSel\_test.go'', analyze its traces for deadlocks, measure the execution coverage and generate visualizations, one can simply execute \textsf{goat -path=CodeBenchmark/goBench\_goat/defSel}. The outputs will be generated automatically in the \goat's workstation directory.

You can add delay(s) around concurrency usages of the target code via \texttt{-d} option and define the ferequency of test executions by \texttt{-freq}

\subsection{Evaluation and expected results}
Obtaining the results presented in this paper requires installation of the competent, dynamic tools and the GoBench bug benchmark:

\begin{listing}[H]
\begin{minted}
[
fontsize=\footnotesize,
linenos=false,
escapeinside=||,
breaklines
]
{bash}
# temporarily switch back to original Go
ln -nsf /usr/local/go-orig /usr/local/go
go get -u go.uber.org/goleak
go get golang.org/x/tools/cmd/goimports
go get github.com/sasha-s/go-deadlock
# switch to GoAT's Go
ln -nsf /usr/local/go-goat/go /usr/local/go

git clone https://github.com/goodmorning-coder/gobench.git
\end{minted}
\end{listing}

Then by executing \textsf{goat -conf=configs/conf\_attn\_blocking\_all -freq=1000 -cov}, table \ref{tab:comparison} is fully constructible.

%%%%%%%%%%%%%%%%%%%%%%%%%%%%%%%%%%%%%%%%%%%%%%%%%%%%%%%%%%%%%%%%%%%%%
\subsection{Notes}
We are actively working on improving \goat and its documentation. Soon we will be having a docker version for \goat. Updates will be reflected in our Github repo and also the submission \cite{goat_artifact}.
