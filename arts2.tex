\subsection{Installation}

GoAT is working in a custom runtime based on version [1.15.6](https://github.com/golang/go/tree/go1.15.6) of Golang.
Lets say the original Go installation is under \texttt{/usr/local/go}. Then the following statements must be executed to build the \goat's runtime and resolve its dependencies:
\begin{listing}[]


\begin{minted}
[
fontsize=\footnotesize,
linenos=false,
escapeinside=||,
breaklines
]
{bash}
sudo -i
mv /usr/local/go /usr/local/go-orig
ln -nsf /usr/local/go-orig /usr/local/go

# download fresh version 1.15.6
wget https://golang.org/dl/go1.15.6.linux-amd64.tar.gz
mkdir -p /usr/local/go-goat
tar -xzvf go1.15.6.linux-amd64.tar.gz -C /usr/local/go-goat

# set environment variables
export GOPATH=$HOME/gopath
mkdir $GOPATH
export GOROOT=/usr/local/go
export PATH=$PATH:$GOROOT/bin:$GOPATH/bin
export GOATWS=$HOME/goatws
mkdir -p $GOATWS
export GOATTO=30
export GOATMAXPROCS=4

# path safety
go get github.com/staheri/goat

# build the runtime
cd /usr/local/go-goat/go
patch -p2 -d src/  < $GOPATH/src/github.com/staheri/goat/go1.15.6_goat_june15.patch
cd src/
export GOROOT_BOOTSTRAP=/usr/local/go-orig
./make.bash
ln -nsf /usr/local/go-goat/go /usr/local/go
cd $GOPATH/src/github.com/staheri/goat
go build -o $GOPATH/bin/goat
\end{minted}
\end{listing}

\subsection{Experiment workflow}

The \goat's help message illustrate its workflow:
\begin{listing}[]
\begin{minted}
[
fontsize=\footnotesize,
linenos=false,
escapeinside=||,
breaklines
]
{bash}

Usage of bin/goat-single:
  -cov
        Include coverage report in evaluation
  -d int
        Number of delays
  -eval_conf string
        Config file with benchmark paths in it
  -freq int
        Frequency of executions (default 1)
  -path string
        Target folder
  -race
        Enable race detection
\end{minted}
\end{listing}

For example, for collecting traces from the execution of \texttt{CodeBenchmark/goBench\_goat/defSel/goatDefSel\_test.go}, analyze its traces for deadlocks, measure the execution coverage and generate visualizations, one can simply execute \texttt{goat -path=CodeBenchmark/goBench\_goat/defSel}.
The output will be generated in a format like below:
\begin{listing}[]
\begin{minted}
[
fontsize=\footnotesize,
linenos=false,
escapeinside=||,
breaklines
]
{bash}

tree $GOATWS/p8/single_defSel
goatws/p8/single_defSel
└── goat_trace
    ├── bin
    │   └── 939498256trace
    ├── concUsage.json
    ├── out
    │   └── goat_goat_d0.out
    ├── results
    │   └── p8_defSel_goat_d0_T1_hitBug.json
    ├── src
    │   └── goatDefSel_test.go
    ├── traces
    │   └── goat_d0
    │       └── defSel_B0_I0.trace
    ├── traceTimes
    │   └── defSel_B0_I0.time
    └── visual
        ├── SUCC_defSel_B0_I0_fullVis.dot
        ├── SUCC_defSel_B0_I0_fullVis.pdf
        ├── SUCC_defSel_B0_I0_gtree.dot
        ├── SUCC_defSel_B0_I0_gtree.pdf
        ├── SUCC_defSel_B0_I0_minVis.dot
        └── SUCC_defSel_B0_I0_minVis.pdf
\end{minted}
\end{listing}
You can add delay(s) around concurrency usages of the target code via \texttt{-d} option and define the ferequency of test executions by \texttt{-freq}

\subsection{Evaluation and expected results}
Obtaining the results presented in this paper requires installation of the competant dynamic tools and the GoBench bug benchmark:

\begin{listing}[]
\begin{minted}
[
fontsize=\footnotesize,
linenos=false,
escapeinside=||,
breaklines
]
{bash}
# temporarily switch back to original Go
ln -nsf /usr/local/go-orig /usr/local/go
go get -u go.uber.org/goleak
go get golang.org/x/tools/cmd/goimports
go get github.com/sasha-s/go-deadlock
# switch to GoAT's Go
ln -nsf /usr/local/go-goat/go /usr/local/go

git clone https://github.com/goodmorning-coder/gobench.git
\end{minted}
\end{listing}

Then by executing \texttt{goat -conf=configs/conf_attn_blocking_all -freq=1000 -cov}, table \ref{tab:comparison} is fully constructible.

%%%%%%%%%%%%%%%%%%%%%%%%%%%%%%%%%%%%%%%%%%%%%%%%%%%%%%%%%%%%%%%%%%%%%
\subsection{Notes}
We are actively working on improving \goat and its documentations. Soon we will be having a docker version for \goat. Updates will be reflected in our github repo and also the zenodo submission.
